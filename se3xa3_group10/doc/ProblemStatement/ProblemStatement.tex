\documentclass{article}
\usepackage[utf8]{inputenc}
\usepackage{tabularx}
\usepackage{booktabs}
\pagenumbering{gobble}
\usepackage[letterpaper, margin=1.5in]{geometry}

\title{SE 3XA3: Problem Statement}

\author{Team 10, MacLunky
		\\ Albert Zhou, zhouj103
		\\Abeer Al-Yasiri, alyasira
		\\ Niyatha Rangarajan, rangaran
}

\date{February 4, 2021}

\begin{document}

\begin{table}[hp]
\caption{Revision History} \label{TblRevisionHistory}
\begin{tabularx}{\textwidth}{llX}
\toprule
\textbf{Date} & \textbf{Developer(s)} & \textbf{Change}\\
\midrule
January 28, 2021 & Albert, Abeer, Niyatha & Version 0 made\\
\midrule
February 4, 2021 &  Niyatha & Updated stakeholders\\
\midrule
April 12, 2021 &  Albert & Revision 1 check\\
\bottomrule
\end{tabularx}
\end{table}

\newpage

\maketitle

\section{Introduction}
\indent \indent The project will undertake the redevelopment of the Spelunky game using the Python2 open source version. The project will focus on redesigning the graphic features of the game, including the layout and mapping. Furthermore, the redevelopment will incorporate new additional game modes where the player will face new challenges which adds multiple levels to the game’s timeline. This will ensure a better interactive experience to play Spelunky. The project will also re-implement the game in Python3, the newest Python version, to increase accessibility to the new version of the game. Lastly, the game will be redeveloped in an innovative way to be more appealing to the gaming community and released under the name of “MacLunky”. 

\section{Importance}
\indent \indent The problem we are addressing is important as it creates an actual playable version of Spelunky by extending map features and player capabilities. The version we use as a reference has only one functional part and moving onto new levels is not achieved. The project will overcome that so the user is able to play a game that involves a functional interactive “player world”.
This project caters to user customization. The map is considered as a player’s world with objects that will appeal to the user. These objects could be treasures, pathways, etc. The visuals of the map per level can incorporate the stakeholder’s interests and feature modes compatible with the player's gaming skills.

\section{Context}
\indent \indent The stakeholders include fans, the developer of the original Spelunky, the target user base, the course instructor and the teaching assistants (TAs) of the course. The fans and developer would be interested in how the game can be implemented in Python instead of GameMaker Studio. The target audience are users aged 10+ who are interested in a 2D cave exploring adventure.The course instructor and the TAs will be setting deadlines for project milestones and provide a view of customer feedback for project submissions. The system is a desktop/laptop application for individual users running on the Python virtual machine. Any machine with Python and Pygame installed can run the system.
If the system is compiled into an executable file, it will be able to run without Python or Pygame.

\end{document}